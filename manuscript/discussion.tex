We have proposed a novel dimensionality reduction technique for use with high-dimensional biological data: \textit{sparse contrastive principal component analysis}. A central contribution of the proposed method is the incorporation of a penalization step that ensures both the sparsity and stability of the principal components generated by contrastive PCA. Our approach allows for high-dimensional biological datasets, such as those produced by the currently popular scRNA-seq experimental paradigm, to be examined in a manner that uncovers interpretable as well as relevant biological signal after the removal of unwanted technical variation, all the while placing only minimal assumptions on the data-generating process.

We also present a data-adaptive and algorithmic framework for applying contrastive dimensionality reduction techniques, like cPCA and scPCA, to high-dimensional biological data. Where the original proposal of cPCA relied upon visual inspection by the user in selecting the ``best'' contrastive parameter \citep{Abid2018}, a notoriously unreliable process, our extension formalizes the data-adaptive selection of tuning parameters. The automation of this step translates directly to significantly increased computational reproducibility. We have proposed the use of cross-validation to select tuning parameters from among a pre-specified set in a generalizable manner, using average silhouette width to assess clustering strength. Several other approaches to the selection of tuning parameters, including the choice of criterion for assessing the ``goodness'' of the dimensionality reduction (here, clustering strength as measured by the average silhouette width), may outperform our approach in practice and could be incorporated into the modular framework of scPCA --- we leave the development of such approaches and assessments of their potential advantages as an avenue for future investigation.

We have demonstrated the utility of scPCA relative to competing approaches, including standard PCA, cPCA, t-SNE, UMAP, and ZINB-WaVE (where appropriate), using both a simulation study of single-cell RNA-seq data and the re-analysis of several publicly available datasets from a variety of high-dimensional biological assays. We have shown that scPCA recovers low-dimensional embeddings similar to cPCA, but with a more easily interpretable principal component structure, and in simulations, diminished technical noise. Further, our results indicate that scPCA generally produces denser, more relevant clusters than t-SNE, UMAP, and ZINB-WaVE. Moreover, we verify that clusters derived from scPCA correspond to biological signal of interest. Finally, as the cost of producing high-dimensional biological data with high-throughput experiments continues to decrease, we expect that the availability and utility of techniques like scPCA --- for reliably extracting rich, sparse biological signals while data-adaptively removing technical artifacts --- will strongly motivate the collection of control samples as a part of standard practice.