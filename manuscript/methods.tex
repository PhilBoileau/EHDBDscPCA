\subsection{Sparse Contrastive PCA}

The sparse contrastive PCA (scPCA) procedure applies SPCA with minimal
modifications \edit{to a pair of target and background datasets' contrastive 
covariance matrix $\mathbf{C}_\gamma$}. The numerical solution to the SPCA 
criterion of Equation \eqref{eq:spca} is obtained by the following
alternating algorithm until convergence of the sparse loadings \citep{Zou2006},
\edit{where $\mathbf{A}_{p\times k}$ is initialized as the matrix of loadings 
corresponding to the $k$ leading principal components of $\mathbf{C}_\gamma$:}

\textbf{For fixed} $\mathbf{A}$:
\edit{Relying on the results of Equation \eqref{eq:cov_mat_spca}, the
elastic net solution for the $j^{\text{th}}$ loading vector is
\begin{equation*}
  \beta_j^\star = \argmin_{\beta_j} \lVert \mathbf{C}_\gamma^{\frac{1}{2}}\alpha_j - \mathbf{C}_\gamma^{\frac{1}{2}}
  \beta_j \rVert_2^2 + \lambda_0 \lVert \beta_j \rVert_2^2 + \lambda_{1, j}
  \lVert \beta_j \rVert_1.
\end{equation*}}
Generally, for ease of computation, $\lambda_{1, j} = \lambda_1$, for
$j=1, \ldots, k$. The entries of the loading matrix $\mathbf{B}$ are independent of
the choice for the $\ell_2$ penalty (ridge) parameter $\lambda_0$
\citep{Zou2006}, which is used only for numerical reasons. The ridge penalty is set to zero when
$\mathbf{C}^{\frac{1}{2}}_\gamma$ is of full rank; otherwise, a small constant
value is used to remedy issues of indeterminacy that arise when fitting the
elastic net.

\textbf{For fixed} $\mathbf{B}$: Only the first term of the SPCA criterion of
Equation \eqref{eq:spca} must be minimized with respect to $\mathbf{A}$. The
solution is given by the reduced rank form of the Procrustes rotation, computed
as $\mathbf{A}^\star = \mathbf{U}\mathbf{V}^\top$ \citep{Zou2006}. The matrices
of left and right singular vectors are obtained from the following singular
value decomposition:
\begin{equation*}
  \mathbf{C}_\gamma\mathbf{B} = \mathbf{U}\mathbf{D}\mathbf{V}^\top.
\end{equation*}
Generally, $\mathbf{C}_\gamma$ is not positive-semidefinite and its square root
is undefined. Instead, a positive-semidefinite matrix
$\widetilde{\mathbf{C}}_\gamma$, approximating $\mathbf{C}_\gamma$, is used.
$\widetilde{\mathbf{C}}_\gamma$ is obtained by replacing the diagonal matrix in
the eigendecomposition of $\mathbf{C}_\gamma$ by a diagonal matrix in which
negative eigenvalues are replaced by zeros \citep{elasticnet}:
\begin{align*}
  \mathbf{C}_\gamma &= \mathbf{V}\mathbf{\Lambda}\mathbf{V}^\top \\
  \widetilde{\mathbf{C}}_\gamma &= \mathbf{V}\mathbf{D}\mathbf{V}^\top \\
\end{align*}
\begin{equation*}
\text{where } \text{D}_{ii} =
  \begin{cases}
    \Lambda_{ii}, & \text{if $\Lambda_{ii} > 0$} \\
    0, & \text{otherwise}
  \end{cases},
  \qquad 
  \text{for } i = 1, \ldots, p.
\end{equation*}

Thus, the directions of variation given by the negative eigenvalues of
$\mathbf{C}_\gamma$ are discarded, as they correspond to those which are dominated by the variance in the background dataset. This procedure can be viewed as a preliminary thresholding of the eigenvectors of $\mathbf{C}_\gamma$,
where the cutoff is an additional hyperparameter corresponding to a non-negative
real number. Explicitly defining a small positive threshold may prove useful for
datasets that possess many eigenvalues near zero, which correspond to sources of
technical and biological noise remaining after the contrastive step.
Empirically, however, providing a wide range of contrastive parameters $\gamma$ has been found to have a similar effect as using
multiple cutoff values, that is, larger values of $\gamma$ naturally produce
sparser matrices $\widetilde{\mathbf{C}}_{\gamma}$.

For the purpose of contrastive analysis, a direction's importance is
characterized by its target-background variance coupling; higher target variance
and lower background variance pairs produce the best directions \citep{Abid2018}
and correspond to the largest positive eigenvalues. The elimination of directions with
negative eigenvalues therefore guarantees that the sparse contrastive PCs (scPCs) are
rotations of the target data relying on the sparse directions most
variable in the target data but least variable in the background data, making
a cutoff of zero a natural choice for the thresholding operation.

\subsection{Framework for Hyperparameter Tuning}\label{hyp_tune}

The scPCA algorithm relies on two hyperparameters: the contrastive parameter $\gamma$ and the $\ell_1$ penalty parameter $\lambda_1$. To select the optimal combination of $\gamma$ and $\lambda_1$ from a grid of \textit{a priori} specified values, we propose to cluster the $n$ observations of the target dataset based on their first $k$ scPCs, selecting as optimal the combination $\{\gamma, \lambda_1\}$ producing the ``strongest'' cluster assignments. This framework casts the selection of $\{\gamma, \lambda_1\}$ in terms of a choice of clustering algorithm, distance metric (based on $\widetilde{\mathbf{C}}_{\gamma}$), and clustering strength criterion. For ease of application, we propose to select $\{\gamma, \lambda_1\}$ by maximization of the average silhouette width over clusterings of the reduced-dimension representation of the target data. This procedure implicitly requires the choice of a clustering algorithm, such as $k$-means~\citep{kmeans}, to be applied to the representation of the data in the first $k$ scPCs. Such methods require an appropriate choice for the number of clusters, which we contend will generally not be a limiting factor in the use of scPCA. Indeed, reasonable choices for the number of clusters can often be inferred in \textit{omics} settings from sample annotation variables accompanying the data or from previously available biological knowledge. In Section~\ref{results}, we empirically demonstrate that the results of the algorithm are robust to the choice of the number of clusters. Additionally, scPCA has no particular dependence on average silhouette width as a criterion, that is, alternative criteria for assessing clustering strength could be used when appropriate. %(e.g., when the data contains sub-clusters~\citep{Liu2010}).
Naturally, this proposed hyperparameter tuning approach can be applied to cPCA by setting $\lambda_1$ to zero.

To address concerns of overfitting and to avoid discovering non-generalizable patterns from the data, we propose the use of cross-validation. For a grid of \textit{a priori} specified contrastive parameters $\gamma$ and $\ell_1$ penalty parameters $\lambda_1$, \edit{$W$-fold} cross-validation may be performed as follows:
\begin{enumerate}
  \itemsep0pt
  \item Partition each of the target and     
     background datasets into \edit{$W$} roughly
     equally-sized subsets.
   \item Randomly pair each of the target \edit{$W$} subsets with one of the
     background subsets; these pairs form the fold-specific \edit{validation sets}.
   \item Iteratively perform scPCA over the 
     observations of the target and
     background data not contained in the validation set (i.e., the training sets),
     for each pair of contrastive parameters and $\ell_1$ penalty parameters in
     the hyperparameter grid.
  \item Project the target validation data onto the low-dimensional space using
     the loading matrices obtained in the prior step.
  \item Compute a clustering strength criterion (e.g., average
  silhouette width) for a clustering of the target validation data with the \textit{a priori}
  specified number of clusters.
  \item Finally, compute the cross-validated average of the clustering
  strength criteria (e.g., cross-validated average of average silhouette width)
  across the validation sets for each pair of hyperparameters, selecting the
  pair that maximizes the value of the criterion.
\end{enumerate}

\edit{We note two caveats of this cross-validation framework: (1) it increases the computation time, and (2) each fold is assumed to be representative of the process which generated the data. For large datasets and say 5 to 10 folds, (2) is not a grave concern; however, as the number of samples decreases, this assumption becomes less tenable. In such a situation, the non-cross-validated framework should be used, or the number of folds reduced. Examples and results are provided in Section~\ref{cv_algo_example}.}

\subsection{Algorithm and Software Implementation}

\edit{The implementation of the scPCA algorithm is presented in
Algorithm~\ref{algo1} of the supplement. Algorithm~\ref{algo2},
also in the supplement, details the cross-validated variant.}

\edit{Regarding the computational complexity of scPCA (and cPCA), we note that
the number of observations in the target and background data are not limiting
factors. Indeed, these algorithms are applied to a $p \times p$
contrastive covariance matrix, a function of the target and background covariance
matrices. The computation time of these covariance matrices increases linearly in
the number of observations $n$ and quadratically in the number of features $p$. The
methods' computational efficiency are therefore most impacted by the number of
features and the size of the hyperparameter grid. A note on cPCA's and
scPCA's running time is provided in Section~\ref{run_time}, as is a comparison
to competing methods.}

A free and open-source software implementation of scPCA is
available in the \texttt{scPCA} package for the \texttt{R} language and
environment for statistical computing~\citep{R}. The \texttt{scPCA} package
\edit{has been released} as part of the Bioconductor Project
\citep{gentleman2004bioconductor,gentleman2006bioinformatics,huber2015orchestrating} (\url{https://bioconductor.org/packages/scPCA}).

The code and data used to generate this manuscript are publicly available on
GitHub (\url{https://github.com/PhilBoileau/EHDBDscPCA}).